\documentclass{ctexart}
\begin{document}
	\title{基于AIoT的智慧消防系统}
	\section{项目方案概述}
	我国当前智能化消防安全技术远远没有跟上城市急速发展的脚步,因此,我们使用NB-IoT、无线感知、目标检测等技术,将物联网与人工智能、大数据相结合,构建一个系统的智能消防平台。在感知层部署烟雾、温湿度等传感器和摄像头,传感器通过NB-IoT协议接入核心网,向云平台上传感知数据。摄像头通过Wifi或有线的方式入网,向云平台上传视频数据。云平台通过分析多源数据,实现火灾预警、火源识别,并通过视频分析紧急出口的人流量,实现人员疏散的智能决策。在救援阶段,我们引入了无线感知技术,能够(待补充)。
	\section{项目团队}
	\section{项目产品化}
		\subsection{项目产品特性}
			\subsubsection{基于NB-IoT的温湿度、烟雾感知}
			NB-IoT,全称为Narrow Band Internet of Things,中文名窄带物联网,是一个为万物互联打造的新的低功率广域网络。顾名思义,NB-IoT所占用的带宽很窄,只需约180KHz,而且其使用License频段,可采取带内、保护带或独立载波三种部署方式,与现有网络共存,并且能够直接部署在GSM、UMTS或LTE网络,即2/3/4G的网络上,实现现有网络的复用,降低部署成本,实现平滑升级。NB-IoT主要具备以下四大特点:1.广覆盖:相比现有的GSM、宽带LTE等网络覆盖增强了20dB,信号的传输覆盖范围更大(GSM基站目前理想状况下能覆盖35km),能覆盖到深层地下GSM网络无法覆盖到的地方。2.大连接:相比现有无线技术,同一基站下增多了50-100倍的接入数,每小区可以达到50K连接,真是实现万物互联所必须的海量连接。3.低功耗:终端在大部分时间内均处于休眠状态,并集成多种节电技术,待机时间最长可达10年。4.低成本:单NB-IoT通信模块成本不足5美元。因此,我们选择NB-IoT作为通信方式。\par
			我们主要使用了DHT11和MQ-2传感器,DHT11负责采集温湿度、MQ-2负责采集烟雾浓度。
			\subsubsection{基于深度学习的火焰识别和人流量监测}
	\section{产品化实施计划}
\end{document}